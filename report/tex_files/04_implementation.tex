%!TEX root=../main.tex

\subsection{Parameter Choices}

In our implementation, we follow \cite{C:CosLonNae16} and choose
\begin{align*}
    p = 2^{372} \cdot 3^{239} - 1.
\end{align*}

Although $p$ can be arbitrary for our protocol, the implementation of the protocol benefits from some special forms of $p$. Choosing $p \equiv -1 \pmod 4$ means that $-1$ is not a square in $\F_{p^2}$ and hence $\F_{p^2} = \F_p(i)$ for $i^2 = -1$, which makes the arithmetic simpler.


% \paragraph{TODO} Write something about the balance of $2^{372} \approx 3^{239}$.

% \paragraph{TODO} Write something about the choice of the curve and etc.

\subsection{Implementation Details}

We use the GMP library \cite{Granlund12} as our high precision arithmetic library. Based on GMP library, we build up several C++ classes and functions based on our purpose.

\begin{itemize}
    \item Integer class: We wrap the GMP library into the integer class, which is more convenient to use.
    \item Fp class: We implement the basic $\F_p$ field using Integer class as well as computing the square root of an element.
    \item Fp2 class: Similar to $\F_p$, we implement the basic $\F_{p^2}$ field containing the square root algorithm.
    \item Curve class and RPoint class: We store the Weierstrass Form of the curve. Both curves and points are stored in projective coordinates to avoid finite field invert operation. The addition and doubling refer to \cite{EFD}.
    \item Isogeny class: The isogenies are computed using Velu's Formula.
    \item IsogenyChain class: The IsogenyChain class maintains compositions of 2-degree or 3-degree isogenies, and use them to construct long isogenies with smooth degree. The construction is achieved by a divide-and-conquer algorithm \cite{PQCRYPTO:JaoDeFo11} on the triangle of isogeny chains. The algorithm takes time complexity of $\tilde{O}(\max\{e_2,e_3\})$.
    \item run\_protocol() in Protocol.cc: We implement the SIDH key exchange protocol followed the algorithms described in the previous sections using the classes and tools above. Finally both Alice and Bob agreed on the same key.
\end{itemize}

\subsection{Experiments}

% Let $e_2 = 372$ and $e_3 = 239$, our program successfully 
We run the test on our laptops. The result is shown below.
\begin{table}[H]
    \centering
    \begin{tabular}{|c|c|}
    \hline
                     & 768-bit $p$  \\ \hline
    Public parameter & 2.05125s  \\ \hline
    Alice round 1    & 0.943308s \\ \hline
    Bob round 1      & 0.90422s  \\ \hline
    Exchange message & 0.393379s \\ \hline
    Alice round 2    & 0.951511s \\ \hline
    Bob round 2      & 0.968126s \\ \hline
    \end{tabular}
    \end{table}