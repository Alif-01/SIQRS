%!TEX root=../main.tex

The Diffie-Hellman scheme is a fundamental protocol for public-key exchange, and many protocols are developed based on DH, such as ECDH (\textit{Elliptic-curve Diffie-Hellman}). However, it is well-known that Shor's algorithm would solve the discrete-log algorithms and break the protocol on a quantum computer. In recent years, D. Jao and L. De Feo \cite{PQCRYPTO:JaoDeFo11} introduced the protocol of SIDH (\textit{Supersingular Isogeny Diffie-Hellman}), which overcomes the previous protocols in a variety of aspects. The SIDH protocol involves new computational assumptions which are not only believed to be more secure but also requires fully exponential time to attack even on quantum computers at current. Besides, the SIDH protocol is also performs faster and easier to implement.

In this report, we delve into the mathematics behind the elliptic curves in Section 2. We then explain the implementation and the security proof of the protocol in Section 3. Finally, we implement the SIDH protocol in C++ and successfully exchange the key between Alice and Bob in Section 4. For the implementation details, please refer to the repository \url{https://github.com/Alif-01/SIQRS}, which includes the C++ source code and also the \LaTeX source code of this report.